\documentclass[10pt]{article}
\usepackage[usenames]{color} %used for font color
\usepackage{amssymb} %maths
\usepackage{amsmath} %maths
\usepackage[utf8]{inputenc} %useful to type directly diacritic characters
\usepackage{tikz}

\usetikzlibrary{positioning,chains,decorations.pathreplacing,arrows}
\usepackage{circuitikz}


\definecolor{bluekeywords}{RGB}{0,0,127}
\definecolor{greencomments}{RGB}{0,127,0}
\definecolor{redstrings}{RGB}{127,0,0}
\definecolor{graynumbers}{RGB}{127,127,127}
\definecolor{glasglowblue}{RGB}{0,56,101}

\definecolor{Burgundy}{RGB}{125,34,57}
\definecolor{Leaf}{RGB}{0,132,61}

\usepackage{listings}
\usepackage{lstautogobble}

\lstset{
  language=C,
  columns=fullflexible,
  tabsize=4,
  basicstyle=\ttfamily,
  numbers=left,
  numbersep = 5pt,
  frame=none,
  xleftmargin=1.5cm,
  breaklines=true,
  autogobble=true,
  showstringspaces=false,
  xleftmargin=2em,
  xrightmargin=2em,
  backgroundcolor=\color{subtlegray},
  frame=single,
  framesep=6pt,
  framexleftmargin=6mm,
  keywordstyle=\color{bluekeywords},
  commentstyle=\color{greencomments},
  stringstyle=\color{redstrings},
  numberstyle=\color{graynumbers},
}

\lstdefinestyle{example}{xleftmargin=0cm, xrightmargin=0em}


\usepackage{tcolorbox}
\tcbuselibrary{listings}


\newcommand{\code}{\lstinline}\begin{document}
\tikzset{
  block/.style = {draw, rectangle, align=center, minimum width=2cm, minimum height=3em},
  input/.style = {coordinate}
}


\begin{tikzpicture}[auto, node distance=2.5cm, thick, >=Stealth]
    \coordinate [input, name=input] (INPUT);
    \node [block, right=of INPUT] (CPP1) {C Preprocessor};
    \node [block, right=of CPP1] (CLANG1) {Compiler};
    \node [block, right=of CLANG1] (AS1) {Assembler};
    \node [output, right=of AS1, name=output] (OUTPUT) {OUT};

    \draw[->] (INPUT) -- node[above,minimum width=2.5cm,align=left,text width=2.5cm]{\code|input.c|} (CPP1);

    \path (INPUT) -- ++(0, -3\baselineskip) coordinate(HEADER11) -- (CPP1.south) -- ++(-0.75, 0) coordinate(HEADER1in);

    \draw[->] (HEADER11) -- node[above,minimum width=2.5cm,align=left,text width=2.5cm]{\code|header_1.h|} (HEADER11 -| CPP1.west) -- (HEADER11 -| HEADER1in) -- (HEADER1in |- CPP1.south);

    \path (INPUT) -- ++(0, -4.5\baselineskip) coordinate(HEADER1dot) -- (CPP1.south) coordinate(HEADER1dotin);

    \path (HEADER1dot) -- node[above,draw=none]{$\vdots$} (HEADER1dot -| CPP1.west) -- (HEADER1dot -| HEADER1dotin) -- (HEADER1dotin |- CPP1.south);

    \draw (CPP1.south) node[below=6pt]{$\dotsb$};

    \path (INPUT) -- ++(0, -6\baselineskip) coordinate(HEADER1n) -- (CPP1.south) -- ++(0.75, 0) coordinate(HEADER1nin);

    \draw[->] (HEADER1n) -- node[above,minimum width=2.5cm,align=left,text width=2.5cm]{\code|header_N.h|} (HEADER1n -| CPP1.west) -- (HEADER1n -| HEADER1nin) -- (HEADER1nin |- CPP1.south);

    \draw[->] (CPP1.east) -- node[above]{\code|input.E|} (CLANG1.west);

    \draw[->] (CLANG1.east) -- node[above]{\code|input.S|} (AS1.west);

    \draw[->] (AS1.east) -- node[above]{\code|output.o|} (OUTPUT);
  \end{tikzpicture}
\end{document}